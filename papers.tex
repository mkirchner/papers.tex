\documentclass{tufte-handout}

\title{Paper notes}

\author{The Author}

\date{}

%\date{28 March 2010} % without \date command, current date is supplied

%\geometry{showframe} % display margins for debugging page layout

\usepackage{graphicx} % allow embedded images
  \setkeys{Gin}{width=\linewidth,totalheight=\textheight,keepaspectratio}
  \graphicspath{{img/}} % set of paths to search for images
\usepackage{amsmath}  % extended mathematics
\usepackage{booktabs} % book-quality tables
\usepackage{units}    % non-stacked fractions and better unit spacing
\usepackage{multicol} % multiple column layout facilities
\usepackage{lipsum}   % filler text
\usepackage{fancyvrb} % extended verbatim environments
  \fvset{fontsize=\normalsize}% default font size for fancy-verbatim environments
\usepackage[ruled]{algorithm2e}

\begin{document}

\maketitle

\begin{abstract}
\noindent
This is a template for paper collection reading notes.
\end{abstract}

%\printclassoptions

\section{Section}\label{sec:example}
\newthought{Introduction to} whatever the set of following papers is about.
\subsection{Paper 1}\label{sec:paper-1}
Here is the summary for the first paper \cite{Tufte2006}. Lorem Ipsum
\footnote{Footnotes have a number and become sidenotes}, more lorem ipsum.
\marginnote{This is a margin note.  Notice that there isn't a number preceding the note, and
there is no number in the main text where this note was written.}

We also add an algorithm here, see Algorithm~\ref{alg:example}.

\begin{algorithm}[tbp]
	\label{alg:example} 
\caption{How to write algorithms}
% \SetAlgoLined
\KwResult{This is the result}
 initialization\;
 \While{condition}{
  instructions\;
  \eIf{condition}{
   instructions1\;
   instructions2\;
   }{
   instructions3\;
  }
 }
\end{algorithm}

\subsection{References}
References are placed alongside their citations as sidenotes,
as well.  This can be accomplished using the normal \Verb|\cite|
command.\sidenote{The first paragraph of this document includes a citation.}

The complete list of references may also be printed automatically by using
the \Verb|\bibliography| command.  (See the end of this document for an
example.)  If you do not want to print a bibliography at the end of your
document, use the \Verb|\nobibliography| command in its place.  

To enter multiple citations at one location,\cite{Tufte2006,Tufte1990} you can
provide a list of keys separated by commas and the same optional vertical
offset argument: \Verb|\cite{Tufte2006,Tufte1990}|.  

\bibliography{papers}
\bibliographystyle{plainnat}



\end{document}
